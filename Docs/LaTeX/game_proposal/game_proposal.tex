\documentclass[11pt,a4paper]{article}
\usepackage[pdftex]{graphicx}
\usepackage{tikz}
\usetikzlibrary{shapes,arrows}
\newcommand{\HRule}{\rule{\linewidth}{0.5mm}}
\begin{document}
\begin{titlepage}
\begin{center}
\textsc{\LARGE StickMatch}\\[1.5cm]

\HRule \\[0.4cm]
{ \huge \bfseries Game Proposal \\[0.4cm] }
\HRule \\[1.5cm]

\begin{minipage}{0.4\textwidth}
\begin{center} \large
\emph{Rico}\\
\end{center}
\end{minipage}

\vfill

{\large \today}
\end{center}
\end{titlepage}
\pagenumbering{Roman}
\tableofcontents
\newpage
\pagenumbering{arabic}
\section*{Overview \\ \HRule \\[0.4cm] }
\addcontentsline{toc}{section}{Overview}
StickMatch is a 3D fighting game for {\it Linux}, {\it Windows},
{\it Dreamcast}, {\it Xbox}, and {\it Pandora}.
The goal of the game is to Knock Out [KO] your opponent via
a series of punches and kicks.  Initially, there will be a single player mode
where the player must defeat eight opponents to win the game.  Each opponent
has a separate {\it stage} which will be themed differently depending on the
character the player is facing.  To advance to the next stage, players must
complete a set amount of rounds (which can be set by the player), defaulting to
three rounds to win each stage.

Players have three main actions they can use: {\it punch}, {\it kick}, and {\it
guard}.  {\it Puch} is a quick jab with either the left or right arm (this is
determined by the cool-down time which resets the arm to alternate).  {\it
Kick} is a high-reach attack, allowing for more damage to be dealt if landed
successfully.  {\it Guard} blocks attacks and dampens the damage inflicted on
the character.

Multi player mode allows for two people to play against each other in any stage
they see fit.  Whether it's Online or as a local game, the host can choose the
settings for the game (such as the rounds available and stage rotation).
\newpage
\section*{Gameplay}
\addcontentsline{toc}{section}{Gameplay}
\tikzstyle{decision} = [diamond, draw, fill=blue!20, text width=4.5em,
	text badly centered, node distance=3cm, inner sep=0pt]
\tikzstyle{block} = [rectangle, draw, fill=blue!20, text width=5em,
	text centered, rounded corners, minimum height=4em]
\tikzstyle{line} = [draw, -latex']
\tikzstyle{cloud} = [draw, ellipse, fill=red!20, node distance=3cm,
	minimum height=2em]
\tikzstyle{input} = [draw, trapezium, text badly centered, fill=green!20,
	text width=4.5em, minimum height=3em,
	trapezium left angle=60, trapezium right angle=-60]

\centering
\begin{tikzpicture}[node distance = 2cm, auto]
\node [block] (start) {Start};
\node [input, below of=start] (charsel) {Select Character};
\node [block, below of=charsel] (stagen) {Stage {\it n}};
\node [decision, below of=stagen] (win) {Win};
\node [decision, below of=win] (contleft) {Continues Left \textgreater 0};
\node [decision, below of=contleft] (continue) {Continue};
\node [block, right of=win, node distance=4cm] (npp) {{\it n}++};

\path [line] (start) -- (charsel);
\path [line] (charsel) -- (stagen);
\path [line] (stagen) -- (win);
\path [line] (win) -- node {No}(contleft);
\path [line] (win) -- node [yshift=-0.5cm] {Yes}(npp);
\path [line] (contleft) -- node {Yes}(continue);
\path [line] (contleft) -| node[xshift=-1.5cm,yshift=-0.25cm] {No}
	([xshift=0.5cm]npp.east) |-(start);
\path [line] (npp) |- (stagen);
\path [line] (continue) -| node [xshift=1cm,yshift=-0.25cm]{Yes}
	([xshift=-1.5cm]continue.west) |-(stagen);
\path [line] (continue) -| node [xshift=-1.5cm,yshift=-0.25cm]{No}
	([xshift=0.5cm]npp.east) |-(start);

\end{tikzpicture}

\newpage
\end{document}

